\section{Introduction}
In the present article, we replicate the results of Huisman  \&  Weissing, 1999, 
``Biodiversity of plankton by species oscillations and 
chaos''\supercite{1999:Huisman}, an attempt to resolve the ``paradox of 
the plankton''\supercite{1961:Hutchinson} with a nonlinear ordinary differential equation model based on resource competition theory.\\

According to many mathematical models, the number of phytoplankton species in a 
single homogeneous medium cannot exceed the number of separate resources 
available\supercite{1960:Hardin,1973:Phillips,1980:Armstrong}. However it is 
very common to observe more species than easily identifiable resources in real-life conditions. This 
led Hutchinson to formulate the paradox of the plankton\supercite{1961:Hutchinson}. Using 
numerical simulations of their ODE model, Huisman \& Weissing\supercite{1999:Huisman} showed that 
``supersaturated coexistence'' is possible, where more consumer species than resource items coexist through oscillations or chaos.\\

In addition to the replication of the numerical results of Huisman and Weissing, we also present new numerical experiments inspired by two follow-up articles, ``Does ``supersaturated coexistence'' resolve the ``paradox of the plankton'' 
?'' by Schippers et al. 2008\supercite{2008:Schippers}, and ``Towards a solution of the plankton paradox: the  importance of physiology and life history” by Huisman et al. 2008\supercite{2008:Huisman}. 
These seemed to show that supersaturated coexistence might be difficult to obtain outside of the restricted parameter scenarios considered in the original article, but did not consider direct perturbation of intrinsic growth rates, as we do here. 

\section{Model}
We describe below the model of phytoplankton community dynamics of Huisman \& Weissing\supercite{1999:Huisman}.
Let $N_i$ and $R_j$ respectively be the population density of species $i$ and concentration of resource 
$j$, $i\in[\![1,n]\!]$ and $j\in[\![1,k]\!]$ with $n$ and $k$ the number of 
different species and resources. The time derivatives of $N_i$ and $R_j$ are 
given by: \\
\begin{align}
	& \frac{dN_i}{dt}= N_i(\mu_i(R_1,...,R_k)~-~m_i)\\
	& \frac{dR_j}{dt}= D(S_j-R_j) - \sum_{i=1}^n c_{ji} 
\mu_i(R_1,...,R_k)N_i
\end{align}
with parameters
\begin{align*}
& m_i \text{ the mortality rate of species $i$}\\
& D \text{ the system's turnover rate}\\
& S_j \text{ the supply concentration of resource $j$}\\
& c_{ji} \text{ the content of resource $j$ in species $i$}\\
& \mu_i \text{ the growth rate of species $i$, defined using the Monod equation and Liebig's law of minimum: }
\end{align*}
\begin{align}
&\mu_i(R_1,...,R_k)~=~\min_{j\in[\![1,k]\!]}(\frac{r_iR_j}{K_{ji}+R_j}). 
\end{align}
The growth rates are defined using: 
\begin{align*}
&r_i \text{ the maximum growth rate of species $i$}\\
&K_{	ji} \text{ the half-saturation constant for resource $j$ of species $i$.}
\end{align*}
In order to reproduce the results of Huisman \& Weissing (1999), the differential 
equations were integrated using the \texttt{deSolve} package in its 1.32 version in \texttt{R} 4.2.0, using the same parameter sets as in the original article.\\

Multiple simulations are performed and illustrated in the original article as well as here: \\
In Figure \ref{figures:Fig1} a) and b), 3 species are competing for 3 resources, in c), 6 species are competing for 3 resources and in d), 9 species are competing for 3 resources. In Figure \ref{figures:Fig2}, 5 species are competing on 3 resources. In Figure \ref{figures:Fig3} 5 species are competing on 5 resources. In Figure \ref{figures:Fig4}, 12 species are competing for 5 resources. 
In Figure \ref{figures:Fig2} and Figure \ref{figures:Fig1} a), b) and for the 
bifurcation diagram of Figure \ref{figures:Fig3}, all of the species are 
introduced at the same time in the simulation. In Figure \ref{figures:Fig4} and 
Figure \ref{figures:Fig1} c), d), the species were introduced sequentially. 
Huisman et al.\supercite{1999:Huisman} have provided the starting times 
of each species introductions when needed, in addition to the dynamical parameters.\\


As Schippers et al.\supercite{2008:Schippers}, we were wondering how and why the parameter sets were initially chosen, 
and if the results would remain the same for slightly different parameters. Several simulations were made in 
Schippers et al. paper\supercite{2008:Schippers} and Huisman et al.'s response\supercite{2008:Huisman}, in order to evaluate how robust was supersaturated 
coexistence. In the same spirit, we carried out new numerical experiments with a slightly different perspective.\\

We chose to focus on evaluating the robustness of the last simulation of Huisman \& Weissing\supercite{1999:Huisman}, displayed on Figure \ref{figures:Fig4}. We focused on perturbating the growth rate parameter, $r_i$, denoted as $\mu_{\text{max}}$ in the follow-up articles\supercite{2008:Schippers,2008:Huisman}. 
As changing only a single one of the $n$ intrinsic growth rates $r_i$ (as done earlier in the follow-up articles\supercite{2008:Schippers,2008:Huisman}) appeared a little artificial to us, 
we chose to randomly perturb all of the $n~~r_i$ at once, as would typically do an ecosystem-wide perturbation that is not directly related to the modelled resources. 
In a first numerical experiment, we considered the exact same invasion sequence as Huisman \& Weissing\supercite{1999:Huisman}. 
In a second step, we started with the full set of species at once.\\
\\
In order to conduct the numerical experiments proposed earlier, the method used to plot the 
fourth Figure has been reused. The two experiments were conducted 400 times each, 
with as many different parameter sets: the $r_i$ were drawn according to a 
truncated normal distribution. The mean was $\mu=1$ and the variance was 
$\sigma=0.1$, corresponding to CV=10\%. The distribution was truncated using $\mu\pm3\sigma$, in other 
words between $0.7$ and $1.3$.\\~\\ 
\section{Results}
\subsection{Reproduction}
We were able to replicate the four figures of Huisman \& Weissing\supercite{1999:Huisman}, presented below. 
\begin{figure}[H]
\begin{center} 
 \includegraphics[width=1\textwidth]{../Code/Figures/Figure_1.pdf}
  \caption{Oscillations on three resources. a), Time course of the abundances 
of three species competing for three resources. b), The corresponding limit 
cycle. c), Small-amplitude oscillations of six species on three resources. 
d), Large-amplitude oscillations of nine species on three resources.}
  \label{figures:Fig1}
\end{center}
\end{figure}
\begin{figure}[H]
\begin{center} 
 \includegraphics[width=1\textwidth]{../Code/Figures/Figure_2.pdf}
  \caption{Chaos on five resources. a), Time course of the abundances of five 
species competing for five resources. b), The corresponding chaotic attractor. 
The trajectory is plotted for three of the five species, from the period from 
$t= 1,000$ to $t=2,000$ days. c), Time course of total community biomass.}
  \label{figures:Fig2}
\end{center}
\end{figure}
\begin{figure}[H]
\begin{center} 
 \includegraphics[width=1\textwidth]{../Code/Figures/Figure_3.png}
  \caption{Bifurcation diagram, for five species competing for five resources. 
a) Show all of the values of species 1, plotted during the period from t=2,000 
to t=4,000 days, as a function of the half-saturation constant $K_{41}$. Part of a) 
is magnified in b). c) show the local minima and maxima of species 1, plotted 
during the period from t= 2,000 to t=4,000 days, as a function of the 
half-saturation constant $K_{41}$. Part of c) is magnified in d).}
  \label{figures:Fig3}
\end{center}
\end{figure}

The bifurcation diagram caption did not seem to correspond to the actual 
Figure, which seemed to display all of the points of the simulation between 
$t=2,000$ and $t=4,000$ rather that only the local extrema.\\
We have therefore chosen to draw those two options. 
\begin{figure}[H]
\begin{center} 
 \includegraphics[width=0.86\textwidth]{../Code/Figures/Figure_4.pdf}
  \caption{Competitive chaos and the coexistence of 12 species on five 
resources. a), The abundances of species 1-6; b), the abundances of species 7-12.}
  \label{figures:Fig4}
\end{center}
  \end{figure}
\subsection{Experiments}
For the first numerical experiment, introducing species one after the other as in the 
original simulation, the following statistics, displaying the frequencies 
(expressed in \%) of the number of species present at the end of the simulation 
could be obtained: (A species has been considered present if $N_i > 0.001$ at 
the end of the simulation )\\
% latex table generated in R 4.2.1 by xtable 1.8-4 package
% Fri Jul 22 17:21:52 2022
\begin{table}[ht]
\centering
\begin{tabular}{rrrrrrrr}
  \hline
 & 0 species & 1 species & 2 species & 3 species & 4 species & 5 species & 6 species \\ 
  \hline
Probability & 0.00 & 13.68 & 26.56 & 35.01 & 11.27 & 8.85 & 4.02 \\ 
   \hline
\end{tabular}
\end{table}

% latex table generated in R 4.2.1 by xtable 1.8-4 package
% Fri Jul 22 17:21:52 2022
\begin{table}[ht]
\centering
\begin{tabular}{rrrrrrrr}
  \hline
 & 7 species & 8 species & 9 species & 10 species & 11 species & 12 species & Supersaturated \\ 
  \hline
Probability & 0.60 & 0.00 & 0.00 & 0.00 & 0.00 & 0.00 & 4.63 \\ 
   \hline
\end{tabular}
\end{table}

\\
This simulation shows that the persistence of five species and more is 
very unlikely, and supersaturated coexistence may be present for a limited domain of parameter space.\\
%%FB: stopped correction here 26/07/2022
\\
The pattern of extinction and its interaction for the first experiment is shown in the Figure \ref{figures:Figexp1}. Note that 
the subfigure 1) represents the original results with $r_i=1 ~\forall i$. 
\begin{figure}[H]
\begin{center} 
 \includegraphics[width=1\textwidth]{../Code/Figures/Figure_exp1.pdf}
  \caption{Visualisation of 8 different simulations of competitive chaos and 
coexistence of 12 species on five ressources, following the Figure 4 method 
except that each species has a different maximum growth rate $r_i$. Except for the first one, the simulations illustrated were randomly picked among the 500 of the experience. Each couple of subplots is labeled with the number of the corresponding simulation.
a), The abundances of species 1-6 ; b), the abundances of species 7-12.}
  \label{figures:Figexp1}
\end{center}
\end{figure}
We observe contrasted stationary endpoints, possibly with some oscillations or 
chaos during the invasion process. As shown in Figure\ref{figures:Tableexp1}, 
however, most species do not persist.\\
It is also noticable that the 5 species on 5 ressources (before the first 
invasion at $t=1,000$) is sometimes already unstable.\\
\\
For the second experiment, introducing all species at the same time, the 
following statistics could be obtained, displaying the frequencies (expressed in 
\%) of the number of species present at the end of the simulation : (A species 
has been considered present if $N_i > 0.001$ at the end of the simulation) \\
% latex table generated in R 4.2.1 by xtable 1.8-4 package
% Fri Jul 22 17:21:49 2022
\begin{table}[ht]
\centering
\begin{tabular}{rrrrrrrr}
  \hline
Extant species & 0 & 1 & 2 & 3 & 4 & 5 & 6 \\ 
  \hline
Probability & 0.00 & 36.40 & 41.40 & 5.00 & 10.00 & 6.00 & 1.20 \\ 
   \hline
\end{tabular}
\end{table}

% latex table generated in R 4.2.1 by xtable 1.8-4 package
% Fri Jul 22 17:21:49 2022
\begin{table}[ht]
\centering
\begin{tabular}{rrrrrrrr}
  \hline
Extant species & 7 & 8 & 9 & 10 & 11 & 12 & Supersaturated \\ 
  \hline
Probability & 0.00 & 0.00 & 0.00 & 0.00 & 0.00 & 0.00 & 1.20 \\ 
   \hline
\end{tabular}
\end{table}

\\
This simulation shows that the subsitance of five species and more is very 
unlikely and the supersaturated coexistance very rare in the parameter space.\\
\\
The pattern of extinction an its interaction for the second experiment is descirbed in following Figure \ref{figures:Figexp2}, 
note that the subfigure 1) represent the original results with $r_i=1 ~\forall 
i$ : 
\begin{figure}[H]
\begin{center} 
 \includegraphics[width=1\textwidth]{../Code/Figures/Figure_exp2.pdf}
  \caption{Visualisation of 8 different simulations of competitive chaos and 
coexistance of 12 species on five ressources, following the Figure 4 method 
exept that $r_i$ was randomly for each of the species and all species were 
introduced at the same time. Except the first one, the simulations illustrated 
were randomly picked among the 500 of the experience. Each couple of subplots is labeled with the number of the corresponding simulation. a),The abundances of 
species 1-6; b). the abundances of species 7-12.}
  \label{figures:Figexp2}
\end{center}
  \end{figure}
As for the first experiment we observed either stationnary endpoints or 
oscillation but with even less persistent species as shown in 
Figure\ref{figures:Tableexp2}.
\section{Discussion}
We were able to successfully replicate the Figures of the original paper.\\ 
\\
There are minor differences in species dynamics in Figure 4, which are arguably due to differences in numerical integration  during 
the resolution.\\
\\
The experiments showed that slightly changing the values of the $r_i$ parameter almost always 
prevents the coexistence of 12 species on five ressources and mostly prevents a 
supersaturated coexistence. This is true when introducing species sequentially 
as in the original paper as well as all at once. \\
\\
Those results corroborate the results of Schippers et 
al.\supercite{2008:Schippers}, who showed that supersaturated coexistence using chaos or 
oscillations are really unlikely in parameter space and requires a really 
precise -in addition to being uncommon- parameterization. Our results are 
therefore consistent with their conclusion that the claim of solving the paradox 
of plankton might be premature. Additional studies, such as Huisman et al. \supercite{2008:Huisman} paper might however show that other ways to maintain more species exists. 



