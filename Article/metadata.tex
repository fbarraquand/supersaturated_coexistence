% DO NOT EDIT - automatically generated from metadata.yaml

\def \codeURL{https://github.com/fbarraquand/supersaturated_coexistence}
\def \codeDOI{}
\def \codeSWH{}
\def \dataURL{}
\def \dataDOI{}
\def \editorNAME{}
\def \editorORCID{}
\def \reviewerINAME{}
\def \reviewerIORCID{}
\def \reviewerIINAME{}
\def \reviewerIIORCID{}
\def \dateRECEIVED{01 November 2018}
\def \dateACCEPTED{01 December 2023}
\def \datePUBLISHED{01 December 2023}
\def \articleTITLE{[Re] Biodiversity of plankton by species oscillations and chaos}
\def \articleTYPE{Replication}
\def \articleDOMAIN{ecology}
\def \articleBIBLIOGRAPHY{bibliography.bib}
\def \articleYEAR{2023}
\def \reviewURL{}
\def \articleABSTRACT{Huisman and Weissing (1999) propose to solve the paradox of plankton using a resource-based differential equations model demonstrating oscilations and chaos. Here we successfully reproduced their results and figures using R. We added two numerical experiments showing, however, that theses results are not robust to mild pertubations of the intrinsic growth rates.}
\def \replicationCITE{Huisman, J., Weissing, F. Biodiversity of plankton by species oscillations and chaos. Nature 402, 407–410 (1999). https://doi.org/10.1038/46540}
\def \replicationBIB{1999:Huisman}
\def \replicationURL{https://rdcu.be/cR83N}
\def \replicationDOI{10.1038/46540}
\def \contactNAME{Frédéric Barraquand}
\def \contactEMAIL{frederic.barraquand@u-bordeaux.fr}
\def \articleKEYWORDS{ecological modelling, plankton, oscillations, coexistence, R}
\def \journalNAME{ReScience C}
\def \journalVOLUME{4}
\def \journalISSUE{1}
\def \articleNUMBER{}
\def \articleDOI{}
\def \authorsFULL{Guillaume Doyen, Coralie Picoche and Frédéric Barraquand}
\def \authorsABBRV{G. Doyen, C. Picoche and F. Barraquand}
\def \authorsSHORT{Doyen, Picoche and Barraquand}
\title{\articleTITLE}
\date{}
\author[1]{Guillaume Doyen}
\author[1,\orcid{https://orcid.org/0000-0002-0867-2130}]{Coralie Picoche}
\author[1,\orcid{https://orcid.org/0000-0002-4759-0269}]{Frédéric Barraquand}
\affil[1]{Institut de Mathématiques de Bordeaux (UMR 5251), University of Bordeaux, CNRS, Bordeaux INP, Talence, France}
