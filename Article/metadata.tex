% DO NOT EDIT - automatically generated from metadata.yaml

\def \codeURL{https://github.com/fbarraquand/supersaturated_coexistence}
\def \codeDOI{}
\def \codeSWH{}
\def \dataURL{}
\def \dataDOI{}
\def \editorNAME{}
\def \editorORCID{}
\def \reviewerINAME{}
\def \reviewerIORCID{}
\def \reviewerIINAME{}
\def \reviewerIIORCID{}
\def \dateRECEIVED{01 November 2018}
\def \dateACCEPTED{}
\def \datePUBLISHED{}
\def \articleTITLE{[Re] Biodiversity of plankton by species oscillations and chaos}
\def \articleTYPE{Replication}
\def \articleDOMAIN{ecology}
\def \articleBIBLIOGRAPHY{bibliography.bib}
\def \articleYEAR{2022}
\def \reviewURL{}
\def \articleABSTRACT{Huisman and Weissing (1999) propose to solve the paradox of plankton using a resource-based differential equations model demonstrating oscilations and chaos. Here we successfully reproduced their results and figures using R. We added two numerical experiments showing, however, that theses results are not robust to mild pertubations of the intrinsic growth rates.}
\def \replicationCITE{Huisman, J., Weissing, F. Biodiversity of plankton by species oscillations and chaos. Nature 402, 407–410 (1999). https://doi.org/10.1038/46540}
\def \replicationBIB{1999:Huisman}
\def \replicationURL{https://rdcu.be/cR83N}
\def \replicationDOI{10.1038/46540}
\def \contactNAME{Frédéric Barraquand}
\def \contactEMAIL{frederic.barraquand@u-bordeaux.fr}
\def \articleKEYWORDS{rescience c, rescience x, R,ecological modelling,plankton}
\def \journalNAME{ReScience C}
\def \journalVOLUME{4}
\def \journalISSUE{1}
\def \articleNUMBER{}
\def \articleDOI{}
\def \authorsFULL{Frédéric Barraquand, Coralie Picoche and Guillaume Doyen}
\def \authorsABBRV{F. Barraquand, C. Picoche and G. Doyen}
\def \authorsSHORT{Barraquand, Picoche and Doyen}
\title{\articleTITLE}
\date{}
\author[1,\orcid{https://orcid.org/0000-0002-4759-0269}]{Frédéric Barraquand}
\author[1,\orcid{https://orcid.org/0000-0002-0867-2130}]{Coralie Picoche}
\author[1]{Guillaume Doyen}
\affil[1]{Institut de Mathématiques de Bordeaux, CNRS UMR5251, Bordeaux, France}
